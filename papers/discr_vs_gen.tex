
\chapter{Discriminative vs. Generative Models}\label{discriminative_modelinmg}
\section{Overview}
Machine learning models can be classified into two main categories, discriminative and generative models. Simply put, a discriminative model makes predictions based on conditional probability $p\left(y|\bx\right)$ and is used for classification or regression problems. In other words, discriminative models distinguishes the decision boundary between the
classes.  It corresponds to learning parameters that maximize the conditional probability
distribution $p(y|\bx)$. On the contrary, a generative model revolves around the distribution of a data set to return a probability for a given example. Rather than
looking at classes and trying to find something to separate them, it focuses
only on the one class at the time and builds a model what that certain class looks like, then turns attention to the other class. To express it more formally, generative models learn parameters that maximize $p\left(\bx|y \right)$ and $p\left(y\right)$. Since
\begin{align}\label{eq:prob_decompostion}
p\left(\bx,y\right) = p\left(\bx|y\right)\cdot p\left(y\right),
\end{align}
with joint PDF it is possible to generate new $\left\lbrace\bx',y'\right\rbrace$ pairs. In some cases, the use of the second decomposition $p\left(\bx,y\right) = p\left(y|\bx\right)\cdot p\left(\bx\right)$ is also an option.  Note that in an unsupervised setting, the task is reduced to inferring only $p\left(\bx\right)$.
\begin{figure}[h]
	\centering
	\begin{minipage}{.5\textwidth}
		\centering
		\includegraphics[trim = 4cm 8cm 24cm 6cm, clip = true, totalheight=0.26\textheight]{plots/Images/discriminative_model.pdf}
		\captionof{figure}{Discriminative approach. }
		\label{fig:test1}
	\end{minipage}%
	\begin{minipage}{.5\textwidth}
		\centering
		\includegraphics[trim =9.6cm 9.6cm 19cm 4cm, clip = true, totalheight=0.26\textheight]{plots/Images/generative_model.pdf}
		\captionof{figure}{Generative approach.}
		\label{fig:test2}
	\end{minipage}
%\caption{Discriminative and Generative approach.}
\end{figure}
\section{Discriminative Modeling}
In this section, we review the basics of discriminative modeling proposed in \cite{HDGEmain}. Given a data distribution through the probability density $p(\boldsymbol{x})$ and a label distribution with probability density $p(y|\boldsymbol{x})$ containing $C$ categories. In this thesis, we focus on classification problems, where the label $y$ is now a qualitative variable, taking on $C$ possible values and comes from a finite set $\pazocal{C}$.  A classification problem is typically solved using a parametric function $f_{\boldsymbol{\theta}}~:~\mathbb{R}^D~\to~\pazocal{C}$, where $\boldsymbol{\theta}$ denotes the parameters of the model. In practice, the function $f_{\boldsymbol{\theta}}$ is often used in the form of $\mathbb{R}^D \to  \mathbb{R}^C$. This function maps each data point $\boldsymbol{x} \in \mathbb{R}^D$ to $C$ real-valued numbers known as logits. It should be noted that $\mathbb{R}^C$ is allowed here due to the utilization of \emph{one-hot encoding}, which will be explained in Section \ref{OHE}. Logits are used to parameterize a categorical distribution through the function
\begin{equation}\label{softmax}
	q_{\boldsymbol{\theta}}\left(y|\boldsymbol{x}\right) = \frac{\exp\left({f_{\boldsymbol{\theta}}\left(\boldsymbol{x}\right)[y]}\right)}{\sum_{y\in\pazocal{C}}\exp\left({f_{\boldsymbol{\theta}}\left(\boldsymbol{x}\right)[y]}\right)},
\end{equation}
which is known as the Softmax. In other words, the data density $p\left(y\vert \bx\right)$ is modeled by a parameterized family of functions $\left\lbrace q_{\bt}\left(y\vert \bx\right) \vert \bt \in \Theta  \right\rbrace$ and thus $p\left(y\vert\bx\right)$ is assumed to belong to this family.   Note that the convention $f_{\boldsymbol{\theta}}\left(\boldsymbol{x}\right)[y]$ means the element $y^{\mathrm{th}}$ of $f_{\boldsymbol{\theta}}\left(\boldsymbol{x}\right)$. For learning $f_{\boldsymbol{\theta}}$ is usually minimized cross-entropy loss 
\begin{equation}\label{crossentropy}
     \mathrm{CE}\left(\bt\right)=-\mathbb{E}_{p_{\mathrm{data}}\left(y,\bx\right)}\left[\log q_{\boldsymbol{\theta}}\left(y|\boldsymbol{x}\right)\right] \approx -\frac{1}{N}\sum_{i=1}^N\log q_{\boldsymbol{\theta}}\left(y_i|\boldsymbol{x}_i\right).
\end{equation} 
The rationale for this objective comes from minimizing the Kullback-Leibler (KL) divergence with a target distribution $p(y| \boldsymbol{x})$ \cite{KL}. In general, the
KL divergence (or KL distance) from $p(y| \boldsymbol{x})$ to $q_{\boldsymbol{\theta}}\left(y|\boldsymbol{x}\right)$ is defined as
\begin{equation}\label{eq:KLdiv}
D_{\mathrm{KL}} \left(p(y| \boldsymbol{x}) || q_{\boldsymbol{\theta}}\left(y|\boldsymbol{x}\right) \right) = \int p(y| \boldsymbol{x})\log\frac{p(y| \boldsymbol{x})}{q_{\boldsymbol{\theta}}\left(y|\boldsymbol{x}\right)}\d{y} = \mathbb{E}_{p(y| \boldsymbol{x})} \left[\log\frac{p(y| \boldsymbol{x})}{q_{\boldsymbol{\theta}}\left(y|\boldsymbol{x}\right)} \right]
\end{equation}
and has the following properties:
\begin{enumerate}
\item $\KL{p(y| \boldsymbol{x})}{q_{\boldsymbol{\theta}}\left(y|\boldsymbol{x}\right)} \geq 0,$
\item $\KL{p(y| \boldsymbol{x})}{q_{\boldsymbol{\theta}}\left(y|\boldsymbol{x}\right)} = 0$ iff $p(y| \boldsymbol{x}) = q_{\boldsymbol{\theta}}\left(y|\boldsymbol{x}\right)$ almost everywhere,
\item $\KL{p(y| \boldsymbol{x})}{q_{\boldsymbol{\theta}}\left(y|\boldsymbol{x}\right)} \neq \KL{q_{\boldsymbol{\theta}}\left(y|\boldsymbol{x}\right)}{p(y| \boldsymbol{x})}$ and KL divergence does not obey the triangle inequality.
\end{enumerate}
The third property indicates that care is needed in the syntax describing KL divergence. We say that \eqref{eq:KLdiv} is from $p(y| \boldsymbol{x})$ to $q_{\boldsymbol{\theta}}\left(y|\boldsymbol{x}\right)$. Using the logarithmic property, \eqref{eq:KLdiv} can be further rewritten in the form 
\begin{equation}
	 \mathbb{E}_{p(y| \boldsymbol{x})} \left[\log\frac{p(y| \boldsymbol{x})}{q_{\boldsymbol{\theta}}\left(y|\boldsymbol{x}\right)} \right] = 
	 \mathbb{E}_{p(y| \boldsymbol{x})} \left[\log p(y| \boldsymbol{x}) \right] - \mathbb{E}_{p(y| \boldsymbol{x})} \left[\log q_{\boldsymbol{\theta}}\left(y|\boldsymbol{x}\right) \right],
	\end{equation}
where subscript $\boldsymbol{\theta}$ emphasizes that $q_{\boldsymbol{\theta}}\left(y|\boldsymbol{x}\right)$ is the approximative density we get to control. Note that
the first term does not depend on $\bt$ and therefore minimizing either CE or KL divergence is equivalent. Finally, by minimizing with respect to $\bt$ we obtain
\begin{equation}
\min_{\boldsymbol{\theta}} D_{\mathrm{KL}} \left(p(y| \boldsymbol{x}) \Vert q_{\boldsymbol{\theta}}\left(y|\boldsymbol{x}\right) \right) = \min_{\boldsymbol{\theta}} - \E_{p(y| \boldsymbol{x})}\left[ \log q_{\boldsymbol{\theta}}\left(y|\boldsymbol{x}\right)\right].
\end{equation}
For the sake of clarity, the expected value will be discussed. In practice, it is dealt with with discrete data, so the term $\E_{ p(y| \boldsymbol{x})}\left[\log q_{\boldsymbol{\theta}}\left(y|\boldsymbol{x}\right)\right]$ takes the form of
\begin{equation}
    \E_{p(y| \boldsymbol{x})}\left[\log q_{\boldsymbol{\theta}}\left(y|\boldsymbol{x}\right)\right] \approx \sum_{k=1}^C p(y_k| \boldsymbol{x})\log q_{\boldsymbol{\theta}}\left(y_k|\boldsymbol{x}\right).
\end{equation}
This part deserves further discussion for a few reasons:
\begin{itemize}
    \item Maximum likelihood estimation (MLE) of $\bt$ is equivalent to minimizing the KL distance.
    \item One may encounter the concepts of minimization or maximization of CE.
\end{itemize}
To address these reasons, it is necessary to briefly review the MLE. The MLE principle assumes that the most reasonable
values for $\bt$ are those for which the probability of the observed sample is
highest. Since $q_{\boldsymbol{\theta}}\left(y|\boldsymbol{x}\right)$ is model PDF, we have to follow the objective function
\begin{equation}
   \pazocal{L}\left(\bt^{}\right)= \sum_{i=1}^{N}\log q_{\boldsymbol{\theta}}\left(y_i|\boldsymbol{x}_i\right),
\end{equation}
which is up to the factor $-\frac{1}{N}$ same as \eqref{crossentropy}. Clearly, this changes only the value of $\pazocal{L}\left(\bt^{}\right)$, but not the location of the optima, so from an optimization perspective, the distinction is not important. However, the negative sign is obviously important since it is the difference between maximizing and minimizing.
Further optimization of $\pazocal{L}\left(\bt\right)$ gives the point estimate
\begin{align}
    \widehat{\bt}_{\mathrm{ML}} &= \argmax_{\bt} \sum_{i=1}^{N}\log q_{\boldsymbol{\theta}}\left(y_i|\boldsymbol{x}_i\right)\label{eq:argmax}\\ 
    &=\argmin_{\bt} -\sum_{i=1}^{N}\log q_{\boldsymbol{\theta}}\left(y_i|\boldsymbol{x}_i\right)\label{eq:argmin}.
\end{align}
It is more common to minimize a function than to maximize it in practice and therefore log--likelihood function is inverted by adding a negative sign to the front yielding a negative log--likelihood. 
\subsection{One--hot encoding}\label{OHE}
 Machine learning (ML) algorithms can misinterpret the numeric values of labels if there exists a hierarchy between them. One--hot encoding is a very common approach for dealing with this issue, in order to improve the algorithm performance. Each unique category value is transformed into a new column and these dummy variables are then filled with 0 or 1 (0 for FALSE and 1 for TRUE). For the sake of clarity, the transformation of a label encoding into a one--hot encoding is illustrated in the following table \ref{tab:OHE}. 
 
 However, this method has its own downsides. For example, it creates new variables and if there exist many unique category values, the models have to deal with a large number of predictors, leading to the so-called \emph{Big-p problem} \cite{Bigp}. Also, one--hot encoding causes multicolinearity between the individual variables, which may lead to reducing the model's accuracy. 
 \begin{table}[h]
 \centering
 	\begin{tabular}{|l|l|l|}
 		\hline
 		Food Name & Categorical \# & Calories \\ \hline
 		Pizza     & 1              & 266      \\ \hline
 		Hamburger & 2              & 295      \\ \hline
 		Caviar    & 3              & 264      \\ \hline
 	\end{tabular}
 	\quad $\Rightarrow$ \quad
	\begin{tabular}{|l|l|l|l|}
		\hline
		Pizza & Hamburger & Caviar & Calories \\ \hline
		1     & 0         & 0      & 266      \\ \hline
		0     & 1         & 0      & 295      \\ \hline
		0     & 0         & 1      & 264      \\ \hline
	\end{tabular}
	\caption{Transformation of a label encoding (left) to the one--hot encoding (right).}
	\label{tab:OHE}
 \end{table}
\section{Generative Modeling}

\subsection{Variational autoencoder}
Assume that the data $\mathbb{X}$ are generated by some random process, involving and unobserved continuous variable $\boldsymbol{z}$, which will be referenced as a latent variable or code. The objective is once again to find the PDF of the given data in a parametric form $p_{\bt}\left(\bx\right)$. One can choose approximative distribution in the form of
\begin{equation}
p_{\bt}\left(\bx\right) = \int p_{\bt}\left(\bx,  \boldsymbol{z}\right)\d{\boldsymbol{z}} =\int p_{\bt}\left(\bx\vert \boldsymbol{z}\right)p_{\bt}\left(\boldsymbol{z}\right)\d{\boldsymbol{z}},
\end{equation}
but such approximation is very expensive to compute or can be even intractable. Intractability of the $p_{\bt}\left(\bx\right)$ makes posterior PDF $p_{\bt}\left(\boldsymbol{z}\vert\bx\right)$ also intractable.
\subsubsection{Naive approach}
\subsubsection{Variational Bayes}
To solve this issue it is necessary to introduce further approximative posterior distribution $q_{\bphi}\left(\boldsymbol{z}\vert\bx\right) \approx p_{\bt}\left(\boldsymbol{z}\vert\bx\right)$ with parameters $\bphi$, preferably Gaussian. Standard terminology refers to the model $q_{\bphi}\left(\boldsymbol{z}\vert\bx\right)$ as a probabilistic \emph{encoder} and  $p_{\bt}\left(\bx\vert \boldsymbol{z}\right)$ is called a probabilistic \emph{decoder}.
For variational autoencoder (VAE) the idea is to use KL distance from $q_{\bphi}\left(\boldsymbol{z}\vert\bx\right)$ to $p_{\bt}\left(\boldsymbol{z}\vert \bx\right)$, yielding
\begin{equation}\label{eq:VAEloss}
\begin{split}
D_{\mathrm{KL}}\left(q_{\bphi}\left(\boldsymbol{z}\vert \bx \right) \Vert p_{\bt}\left(\boldsymbol{z}\vert \bx\right)\right) & = 
\int q_{\bphi}\left(\boldsymbol{z}\vert \bx \right) \log \frac{q_{\bphi}\left(\boldsymbol{z}\vert \bx \right)}{p_{\bt}\left(\boldsymbol{z}\vert \bx\right)} \d{\boldsymbol{z}} \\
& =  \int q_{\bphi}\left(\boldsymbol{z}\vert \bx \right) \log \frac{q_{\bphi}\left(\boldsymbol{z}\vert \bx \right)p_{\bt}\left(\bx\right)}{p_{\bt}\left(\bx \vert \boldsymbol{z}\right) p_{\bt}\left(\boldsymbol{z}\right)} \d{\boldsymbol{z}} \\
& = \log p_{\bt}\left(\boldsymbol{x}\right) +  \int q_{\bphi}\left(\boldsymbol{z}\vert \bx \right) \log \frac{q_{\bphi}\left(\boldsymbol{z}\vert \bx \right)}{p_{\bt}\left(\bx \vert \boldsymbol{z}\right)p_{\bt}\left(\boldsymbol{z}\right) } \d{\boldsymbol{z}} \\
& = \log p_{\bt}\left(\boldsymbol{x}\right) +  \mathbb{E}_{q_{\bphi}\left(\boldsymbol{z}\vert \bx \right)}\left[\log\frac{q_{\bphi}\left(\boldsymbol{z}\vert \bx \right)}{p_{\bt}\left(\boldsymbol{z}\right)} - \log p\left(\textbf{x}\vert \boldsymbol{z}\right)\right]\\
    & = \log p_{\bt}\left(\boldsymbol{x}\right) +\KL{q_{\bphi}\left(\boldsymbol{z}\vert \bx \right)}{p_{\bt}\left(\boldsymbol{z}\right)} -  \mathbb{E}_{q_{\bphi}\left(\boldsymbol{z}\vert \bx \right)}\left[\log p\left(\bx\vert \boldsymbol{z}\right)\right].
 \end{split}
\end{equation}
Using the last equality of \eqref{eq:VAEloss}, it is possible to rewrite the equation into its typical form
\begin{equation}
\log p_{\bt}\left(\bx\right) - D_{\mathrm{KL}}\left(q_{\bphi}\left(\boldsymbol{z}\vert \bx \right) \Vert p_{\bt}(\boldsymbol{z}\vert \bx)\right) = \underbrace{\mathbb{E}_{q_{\bphi}\left(\boldsymbol{z}\vert \bx \right)}\left[\log p_{\bt}(\bx\vert \boldsymbol{z}) \right] - D_{\mathrm{KL}}\left(q_{\bphi}\left(\boldsymbol{z}\vert\bx \right) \Vert p_{\bt}\left(\boldsymbol{z}\right)\right)}_{\text{= $L\left(\bt,\bphi; \bx\right)$}},
\end{equation}
where the right hand side is called \emph{variational} lower bound. There is no uniformity in terminology and thus one can also encounter the name evidence lower bound (ELBO). The first term of the right hand side is known as reconstruction loss and the second term is often called regularization term. As a KL distance is always non-negative, it holds
\begin{equation}
    \log p_{\bt}\left(\bx\right) \geq L\left(\bt,\bphi; \bx\right).
\end{equation}
The objective is to maximize the log-likelihood $\log p_{\bt}\left(\bx\right)$ which is equivalent to minimizing the negative log-likelihood and that is what will be used here. At this point we have a lower bound for one datapoint $\bx$, but we need to include all observations in the lower bound. Joint log-likelihood can be rewritten as a sum over the marginal log-likelihoods of individual observations $\log p_{\bt}\left(\bx_1,\bx_2,\dots,\bx_N\right)~=~\sum_{i=1}^N\log p_{\bt}\left(\bx_i\right)$ which completes all the building blocks needed to determine the optimization equation. This formulation provides one major advantage, which is that it is now possible to jointly optimize both the generative parameters $\bt$ and the variational parameters $\bphi$ as follows 
\begin{align}
   \widehat{\bt}, \widehat{\bphi} &= \argmin_{\bt,\bphi}-\sum_{i=1}^N\log p_{\bt}\left(\bx_i\right)\\
   &=\argmin_{\bt,\bphi}-\sum_{i=1}^N L\left(\bt,\bphi; \bx_i\right)\\
  &= \argmin_{\bt,\bphi}-\sum_{i=1}^N\mathbb{E}_{q_{\bphi}\left(\boldsymbol{z}\vert \bx_i \right)}\left[\log p_{\bt}(\bx_i\vert \boldsymbol{z}) \right] - D_{\mathrm{KL}}\left(q_{\bphi}\left(\boldsymbol{z}\vert\bx_i \right) \Vert p_{\bt}\left(\boldsymbol{z}\right)\right).\label{eq:VAEloss}
\end{align}
For a better understanding of the problem, a diagram of the VAE is shown in the Figure \ref{fig:VAE_architecture}.
 \begin{figure}[h]
	\centering
	\includegraphics[width=\textwidth, trim={0 18cm 0 0}]{plots/Images/VAE_diagram.pdf}
	\caption{VAE diagram.}%
	\label{fig:VAE_architecture}%
\end{figure}

But before the actual optimization, it is essential to solve the stochastic node problem.   The key success of VAE is in the fact that \eqref{eq:VAEloss} can be efficiently computed using the \emph{reparameterization trick}
\begin{equation}
\boldsymbol{z}_{i,j} = \boldsymbol{\mu}_i + \boldsymbol{\sigma}_i\odot\boldsymbol{\varepsilon}_j ,
\end{equation}
where the symbol $\odot$ denotes Hadamard product, i.e. element-wise product and $\boldsymbol{\varepsilon} \sim \pazocal{N}\left(\boldsymbol{0},\mathbb{I} \right)$. 
Before applying the ELBO loss function to an optimization problem to backpropagate the gradient, it is necessary to make it differentiable by applying the so-called reparameterization trick to remove the stochastic sampling from the formation, and thus making it differentiable. 



\begin{equation}
q_{\bphi}\left(\boldsymbol{z}\vert \bx \right) = \pazocal{N}\left(\boldsymbol{z}; \boldsymbol{\mu},\boldsymbol{\sigma}  \right) ,
\end{equation}
můžeme parametry $\theta$ a $\phi$ minimalizovat zároveň a to následovně
\begin{equation}
\begin{split}
\hat{\theta},\hat{\phi} & = \argmin_{\theta, \phi}\sum_{i=1}^n\log p\left(x_i\right) \\
& = \argmin_{\theta, \phi}\left\lbrace  \mathbb{E}_q\left[\log p(\textbf{x}\vert \textbf{z}) \right] - D_{KL}\left(q\left(\textbf{z}\vert \textbf{x} \right) \Vert p(\textbf{z})\right)\right\rbrace . 
\end{split}
\end{equation}

\begin{equation}
\bx = f_{\bt}(\boldsymbol{z}) + \epsilon ,
\end{equation}
kde $\epsilon \sim \pazocal{N}\left(0,\sigma^2\cdot\mathbb{I}   \right)$ and  $f_{\theta}(\textbf{z})$ je neuronová síť.  Využijeme následující formu aproximace

Podle vztahu pro $\textbf{x}$ určíme distribuce $p\left(\textbf{x}\vert \textbf{z}\right)$ a $p\left(\textbf{z}\right)$ zvolíme jednoduše
\begin{equation}\label{VAE_latent}
\begin{split}
 p(\textbf{x}\vert \textbf{z}) &= \pazocal{N}\left(f_{\theta}(\textbf{z}),\sigma^2\cdot\mathbb{I} \right), \\
p(\textbf{z}) &= \pazocal{N}\left(0,\mathbb{I} \right).
\end{split}
\end{equation}

\subsubsection{Naivní přístup}
K nalezení $p(\textbf{x})$ je třeba najít parametry $\theta$ transformace $f_{\theta}(\textbf{z})$, proto zkusme sestavit věrohodnostní funkci $ \log p\left( \textbf{x}\right) = \log \prod_{i = 1}^n p\left(x_i \right) $  a minimalizovat 
\begin{equation}
\begin{split}
\hat{\theta} & = \argmin_{\theta} \sum_{i=1}^n \log p\left(x_{i} \right)\\
& =  \argmin_{\theta} \sum_{i=1}^n \log \int \pazocal{N}\left(f_{\theta}(z_j),\sigma^2 \right)\cdot\pazocal{N}\left(0,1 \right)    \dd z_j \\
& = \argmin_{\theta} \sum_{i=1}^n \ \log\sum_{j=1}^n \exp \left\lbrace -\frac{1}{2\sigma^2} \left(x_i - f_{\theta}(z_j)  \right)^2 \right\rbrace \cdot \exp \left\lbrace -\frac{z_j^2}{2} \right\rbrace .
\end{split}
\end{equation}
Integrace přes $\textbf{z}$ je nahrazena vzorkováním. Tento postup ovšem při minimalizaci nemusí konvergovat ke správným výsledkům.
\subsection{Variační Bayesova metoda}
Lepší metodou se ukazuje vzorkovat z podmíněné distribuce $q(\textbf{z}\vert \textbf{x})$ a využít ELBO

\begin{equation}
\begin{split}
D_{KL}\left(q\left(\textbf{z}\vert \textbf{x} \right) \Vert p(\textbf{z}\vert \textbf{x})\right) & = 
 \mathbb{E}_q\left[\log q(\textbf{z}\vert \textbf{x}) - \log p\left(\textbf{z}\vert \textbf{x} \right)\right] \\
 & =  \mathbb{E}_q\left[\log q(\textbf{z}\vert \textbf{x}) - \log p(\textbf{x}\vert \textbf{z}) - \log p(\textbf{z}) + \log p(\textbf{x})   \right] .
 \end{split}
\end{equation}
Tuto rovnici můžeme přepsat pomocí KL--divergence 
\begin{equation}
\log p(\textbf{x}) - D_{KL}\left(q\left(\textbf{z}\vert \textbf{x} \right) \Vert p(\textbf{z}\vert \textbf{x})\right) = \mathbb{E}_q\left[\log p(\textbf{x}\vert \textbf{z}) \right] - D_{KL}\left(q\left(\textbf{z}\vert\textbf{x} \right) \Vert p(\textbf{z})\right) ,
\end{equation}
kde pravá strana této rovnice je lower bound objektu $\log p(\textbf{x})$.
Jestliže vybereme parametrickou formu distribuce
\begin{equation}
q\left(\textbf{z}\vert \textbf{x} \right) = \pazocal{N}\left(\mu_{\phi}(\textbf{x}), \diag\left(\sigma^2_{\phi}(\textbf{x})\right) \right) ,
\end{equation}
můžeme parametry $\theta$ a $\phi$ minimalizovat zároveň a to následovně
\begin{equation}
\begin{split}
\hat{\theta},\hat{\phi} & = \argmin_{\theta, \phi}\sum_{i=1}^n\log p\left(x_i\right) \\
& = \argmin_{\theta, \phi}\left\lbrace  \mathbb{E}_q\left[\log p(\textbf{x}\vert \textbf{z}) \right] - D_{KL}\left(q\left(\textbf{z}\vert \textbf{x} \right) \Vert p(\textbf{z})\right)\right\rbrace . 
\end{split}
\end{equation}
V metodě variačního autoencoderu jsou nezbytné následující dva fakty.
\begin{enumerate}
\item Trik v \textbf{reparametrizaci}
\begin{equation}
\textbf{z} = \mu_{\phi}(\textbf{x}) + \sigma_{\phi}(\textbf{x})\odot\epsilon ,
\end{equation}
kde $\odot$ značí Hadamardův součin, čili součin po složkách.
To můžeme zapsat jednodušeji takto
\begin{equation}
z_i = \mu_{\phi}(x_i) + \sigma_{\phi}(x_i)\cdot\epsilon_i .
\end{equation}
Nejedná se v podstatě o nic jiného, než o transformaci náhodné veličiny.
\item  KL--divergence dvou Gaussovských distribucí má analytické řešení a nabude tvaru
\begin{equation}
\begin{split}
 D_{KL}\left(q\left(\textbf{z}\vert \textbf{x} \right) \Vert p(\textbf{z})\right) & = \frac{1}{2}\left[\tr\left(\diag\left(\sigma_{\phi}^2(\textbf{x})\right)\right) - \mu_{\phi}\tran(\textbf{x})\mu_{\phi}(\textbf{x}) - k - \log\det \diag\left(\sigma_{\phi}^2(\textbf{x})   \right)\right] \\
 & = \frac{1}{2}\left[\sum_{l = 1}^k(\sigma_{\phi}^2(\textbf{x})) -\mu_\phi\tran(\textbf{x})\mu_{\phi}(\textbf{x}) - k - \sum_{l = 1}^k\log\sigma_{\phi}^2(\textbf{x})    \right],
\end{split}
\end{equation}
kde $k$ značí dimenzi Gaussova rozdělení. 
\end{enumerate}

Kdybychom totiž nevybrali aproximační distribuce Gaussovské, nemohli bychom tímto způsobem $\hat{\theta},\hat{\phi}$ určit. Díky těmto dvěma faktům tak získáme konečný tvar odhadu parametrů

\begin{multline}\label{řešení_VAE}
 \hat{\theta},\hat{\phi}   = \argmin_{\theta, \phi}\sum_{i = 1}^n\sum_{j = 1}^p \left[ x_i - f_\theta \left(\mu_{\phi}(x_i) + \sigma_{\phi}(x_i)\cdot\epsilon_{i,j}  \right)        \right] ^2  \\ -   \frac{1}{2}\left[\sum_{l = 1}^k(\sigma^2_{\phi}(x_i)) -\mu_{\phi}\tran(x_i)\mu_{\phi}(x_i) - k - \sum_{l = 1}^k\log\sigma^2_{\phi}(x_i)   \right] .
\end{multline}



\subsection{Noise--Contrastive Estimation}
Suppose one has to estimate a model that is specified by an non-normalized probability density function $q^0_{\bt}\left(\boldsymbol{x}\right)$. In such a case, one can utilize noise--contrastive estimation (NCE). The first step is to introduce another parameter $c$ among the estimated parameters $\bt$. For clarity, the symbol $\bt^\star=\left\lbrace\bt^{},c^{}_{}\right\rbrace$ is introduced for the set of estimated parameters, including
$c$. Using this notation, we can write the following equality
\begin{align}
    \log q_{\bt^\star}\left(\boldsymbol{x}\right) = \log q_{\bt^\star}^0\left(\boldsymbol{x}\right) + c,
    \end{align}
which means that the newly introduced parameter $c$ is an estimate of the negative logarithm of
the normalization constant $Z\left(\bt\right)$ \eqref{eq:partitionfunction}.
As the name suggests, we use noise to estimate. By our convention, let $\boldsymbol{X} = \left\lbrace\bx_1,\bx_2,\dots,\bx_N\right\rbrace$
be the observations and $\boldsymbol{\Xi} = \left\lbrace\boldsymbol{\varepsilon}_1,\boldsymbol{\varepsilon}_2,\dots,\boldsymbol{\varepsilon}_N\right\rbrace$ be the artificially generated noise data with known distribution $\psi\left(\boldsymbol{\varepsilon}\right)$. The estimate $\widehat{\bt^\star}$ is then defined as
\begin{align}
    \widehat{\bt^\star} &= \argmax_{\bt^\star} \pazocal{L}^{\mathrm{NC}}\left(\bt^\star\right)\\
   &= \argmax_{\bt^\star} \frac{1}{2N}\sum_{i=1}^N \log S_{\bt^\star}\left(\bx_i\right) + \log\left(1-S_{\bt^\star}\left(\boldsymbol{\varepsilon}_i\right) \right)\label{NCEloss1}\\
   &= \argmin_{\bt^\star} -\frac{1}{2N}\sum_{i=1}^N \log S_{\bt^\star}\left(\bx_i\right) + \log\left(1-S_{\bt^\star}\left(\boldsymbol{\varepsilon}_i\right) \right)\label{NCEloss2}
\end{align}
where $S_{\bt^\star}$ stands for a logistic function,
\begin{align}
S_{\bt^\star}\left(\bx\right) = \frac{1}{1 + \exp\left(-G_{\bt^\star}\left(\bx\right) \right)}
\end{align}
and finally, the function $G_{\bt^\star}$ represents the difference of the log-likelihoods of $q_{\bt^\star}$ and $\psi$, hence 
\begin{align}\label{eq:NCE_G}
    G_{\bt^\star}\left(\bx\right) = \log q_{\bt^\star}\left(\bx \right) - \log\psi\left(\bx \right).
\end{align}
It may be noted that equation \eqref{NCEloss1} also appears in SL tasks and is called binary
CE loss. It is actually a special case of CE itself. Thus, it is used for the classification of two classes. This gives an intuitive insight into how noise--contrastive estimation
really works. When data and noise are compared, the model is learned, so this method can be called
learning by comparison. To make the connection with SL more explicit, denote $U = \left\lbrace\boldsymbol{u}_1, \boldsymbol{u}_2,\dots,\boldsymbol{u}_{2N} \right\rbrace$ the union of two sets $\boldsymbol{X}$ and $\boldsymbol{\Xi}$. Then each data point $\boldsymbol{u}_i$ is assigned a binary class label $y_i$, where $y_i = 1$ if $\boldsymbol{u}_i \in \boldsymbol{X}$ and $y_i = 0$ if $\boldsymbol{u}_i \in \boldsymbol{\Xi}$. The aim is to estimate the posterior probabilities of the classes given the data $\boldsymbol{u}_i$. To do this, one needs the class--conditional PDFs that are given by
\begin{equation}
    p\left(\boldsymbol{u}\vert y =1 \right) =  q_{\bt^\star}\left(\boldsymbol{u} \right) \qquad p\left(\boldsymbol{u}\vert y =0 \right) =  \psi\left(\boldsymbol{u} \right).
\end{equation}
Class labels are equally likely, so that $\mathrm{Pr}\left(y = 1\right) =\mathrm{Pr}\left(y = 0\right)=\frac{1}{2}$ and the posteriors are determined as follows
\begin{align}
    \mathrm{Pr}\left( y=1 \vert\boldsymbol{u} \right) &= \frac{q_{\bt^\star}\left(\boldsymbol{u} \right)}{q_{\bt^\star}\left(\boldsymbol{u} \right) + \psi\left(\boldsymbol{u} \right)} = S_{\bt^\star}\left(\boldsymbol{u}\right),\\
    \mathrm{Pr}\left( y=0\vert \boldsymbol{u} \right) &= 1 - S_{\bt^\star}\left(\boldsymbol{u}\right).
\end{align}
The class labels $y_i$ are Bernoulli--distributed so that
for the log--likelihood of Bernoulli we get
\begin{align}
    \pazocal{L}^{\mathrm{NC}}\left(\bt\right) &= \sum_{i=1}^{2N} y_i\log \mathrm{Pr}\left( y=1 \vert\boldsymbol{u}_i \right) + \left(1-y_i\right)\log \mathrm{Pr}\left( y=0 \vert\boldsymbol{u}_i \right) \\
    &= \sum_{i=1}^{N}\log S_{\bt^\star}\left(\boldsymbol{x}_i\right) +\log \left(1 - S_{\bt^\star}\left(\boldsymbol{\epsilon}_i\right)\right),
\end{align}
which is the equation (up to extrinsic factor $\frac{1}{2N}$) that is optimized in \eqref{NCEloss1} or \eqref{NCEloss2}.
\subsubsection{Choice of the contrastive noise PDF}
The noise distribution $\psi\left(\boldsymbol{\varepsilon}\right)$ can be considered as a design parameter. But this choice is not completely arbitrary, because in practice the noise distribution should meet certain conditions. These are:
\begin{enumerate}
    \item It is easy to sample from, because NCE approach relies on artificially generated noise data $\boldsymbol{\varepsilon}_1,\boldsymbol{\varepsilon}_2,\dots,\boldsymbol{\varepsilon}_N$. 
    \item In order to smoothly evaluate \eqref{eq:NCE_G}, closed form for $\log\psi\left(. \right)$ is requisite.
    \item It leads to a small mean squared error $\mathbb{E}\left[\left(\widehat{\bt^\star} - \bt^\star\right)^2\right]$.
\end{enumerate}
The authors of [] suggest using a Gaussian or uniform distribution, eventually a Gaussian mixture. 
\begin{example}[One--dimensional Gaussian distribution]
\begin{figure}[h]
	\centering
	\subfloat[Loss function minimizing.]
	{{\includegraphics[width=8.0cm]{plots/Images/NCE_loss.pdf} }}%
	\subfloat[Comparison of true (blue) and estimated (red) PDF.]
	{{\includegraphics[width=8.0cm]{plots/Images/NCE_reselts2.pdf} }}%
	\caption{Results of the NCE experiment for one--dimensional Gaussian case.}%
	\label{ex:NCE_1}%
\end{figure}
\end{example}
To test this approach, we performed a simple experiment. There are a total of $N = 100$ i.i.d. and one-dimensional observations $x_1,x_2,\dots,x_N$ from an unknown distribution that is assumed to be non--normalized and Gaussian. Therefore, it is of the form
\begin{align}\label{eq:NCE_datadist1}
    q_{\bt^\star}\left(x\right) = \exp\left(-\frac{1}{2}\cdot\frac{\left(x-\mu\right)^2}{\sigma^2} + c \right),
\end{align} 
where $\bt^\star = \left\lbrace \mu, \sigma^2, c \right\rbrace$. Next, we artificially generate noise data $e_1,e_2,\dots,e_N$, which is again easier to do using a Gaussian distribution. This means that it can be chosen, for example,
\begin{align}\label{eq:NCE_noisedist1}
    \psi\left(e\right) = \frac{1}{\sqrt{2\pi 10}}\exp\left(-\frac{1}{2}\cdot\frac{e^2}{10} \right).
\end{align}
We choose the noise PDF intentionally so widely spread from its mean value because these two PDFs, i.e. \eqref{eq:NCE_datadist1} and \eqref{eq:NCE_noisedist1}, should at least partially overlap. At this point, we have all the components available and it is possible to construct a function $-\pazocal{L}^{\mathrm{NC}}\left(\bt^\star\right)$ that is minimized by using the ADAM optimization algorithm []. The following figure shows the training process and the comparison between
the estimated distribution and the true one. As can be seen in Figure \ref{ex:NCE_1}, this approach works quite well and for more observations, the results would be even better. In addition, the minimization of $-\pazocal{L}^{\mathrm{NC}}\left(\bt^\star\right)$ is very fast.


\begin{example}[Two--dimensional Gaussian distribution]
\begin{figure}[h]
	\centering
	\subfloat[Loss function minimizing.]
	{{\includegraphics[width=8.0cm]{plots/Images/NCE_loss_2D} }}%
	\subfloat[Comparison of true (blue) and estimated (red) PDF.]
	{{\includegraphics[width=8.0cm]{plots/Images/NCE_results_2D.pdf} }}%
	\caption{Results of the NCE experiment for two--dimensional Gaussian case.}%
	\label{ex:NCE_2}%
\end{figure}

The one--dimensional case may seem too simple, and therefore an example with a two-dimensional Gaussian distribution was performed. The experimental setup remains nearly the same; only the dimensionality of the problem differs. 

Recall that the non--normalized multivariate Gaussian distribution in $\R^2$ can be written as
\begin{equation}
   q_{\bt^\star}\left(\bx\right) = \exp\left(-\frac{1}{2}\left(\boldsymbol{x} - \boldsymbol{\mu}\right)^\top\boldsymbol{\Sigma}^{-1}\left(\boldsymbol{x} - \boldsymbol{\mu}\right)+ c \right),
\end{equation}
where $\boldsymbol{\mu} \in \R^2$ and $\boldsymbol{\Sigma}\in \R^{2\times 2}$ is a symmetric and positive semidefinite covariance matrix. As the noise PDF is chosen 
$\psi\left(\boldsymbol{e}\right)=\pazocal{N}\left(\boldsymbol{e}; \boldsymbol{\mu}_1,\boldsymbol{\Sigma}_1\right)$, where $\boldsymbol{\mu}_1 = \left(2,2\right)^\top$ and $\boldsymbol{\Sigma}_1=10\cdot\mathbb{I}_2$. Figure \ref{ex:NCE_2} shows the results in a similar vein to the previous case.
\end{example}
