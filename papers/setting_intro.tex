\def\documentdate{May 2, 2022}

%%\def\documentdate{\today}

\pagestyle{empty}
{\centering

\noindent %
\begin{minipage}[c]{3cm}%
\noindent \begin{center}
\includegraphics[width=3cm,height=3cm,keepaspectratio]{plots/Images/TITLE/cvut}
\par\end{center}%
\end{minipage}%
\begin{minipage}[c]{0.6\linewidth}%
\begin{center}
\textsc{\large{}Czech Technical University in Prague}{\large{}}\\
{\large{}Faculty of Nuclear Sciences and Physical Engineering}
\par\end{center}%
\end{minipage}%
\begin{minipage}[c]{3cm}%
\noindent \begin{center}
\includegraphics[width=3cm,height=3cm,keepaspectratio]{plots/Images/TITLE/fjfi}
\par\end{center}%
\end{minipage}

\vspace{3cm}

\textbf{\huge{}Hybrid Discriminative-Generative Training for Set data}{\huge\par}

\vspace{1cm}

\selectlanguage{czech}%
\textbf{\huge{}Hybridní diskriminativně-generativní modely pro množinová data}{\huge\par}

\selectlanguage{american}%
\vspace{2cm}

{\large{}Master Thesis}{\large\par}

}

\vfill{}

\begin{lyxlist}{MMMMMMMMM}
\begin{singlespace}
\item [{Author:}] \textbf{Bc. Jakub Bureš}
\item [{Supervisor:}] \textbf{doc. Ing. Václav Šmídl, Ph.D.}
\end{singlespace}

\begin{singlespace}
\item [{Academic~year:}] 2021/2022
\end{singlespace}
\end{lyxlist}

~\newpage{}

\noindent \emph{\Large{}Acknowledgment:}{\Large\par}

\noindent I would like to thank doc. Ing. Václav Šmídl, Ph.D.
for his expert guidance and patience during this academic year.

\vfill

\noindent \emph{\Large{}Author's declaration:}{\Large\par}

\noindent I declare that this Master Thesis is entirely
my own work and I have listed all the used sources in the bibliography.

\bigskip{}

\noindent Prague, \documentdate\hfill{}Bc. Jakub Bureš

\vspace{2cm}

\newpage{}

\selectlanguage{czech}%
\begin{onehalfspace}
\noindent \emph{Název práce:}

\noindent \textbf{Hybridní diskriminativně-generativní modely pro množinová data}
\end{onehalfspace}

\bigskip{}

\noindent \emph{Autor:} Bc. Jakub Bureš

\bigskip{}

\noindent \emph{Obor:} Matematické inženýrství\bigskip{}

\noindent \emph{Zaměření:} Aplikované matematicko-stochastické metody

\bigskip{}

\noindent \emph{Druh práce:} Diplomová práce

\bigskip{}

\noindent \emph{Vedoucí práce:} doc. Ing. Václav Šmídl, Ph.D.

\bigskip{}


\noindent \emph{Abstrakt:} Tato diplomová práce se zabývá hybridními diskriminativními a generativními modely a jejich možným využitím v multi--instačním učení, kde je jeden vzorek tvořen množinou vektorů. Nejprve se seznámíme s technikáliemi tohoto přístupu, po té provedeme jednoduchý experiment a nakonec se budeme věnovat samotnému multi--instančnímu učení. Hlavním cílem je pak využít strukturu HMill s knihovnou Mill.jl a rozšírit je o konstrastivní učení na množinová data, kde ukážeme výhody oproti diskriminativnímu učení.

\bigskip{}

\noindent \emph{Klíčová slova:} hybridní diskriminativní a generativní modely, multi--instanční učení

\selectlanguage{american}%
\vfill{}
~

\begin{onehalfspace}
\noindent \emph{Title:}

\noindent \textbf{Hybrid Discriminative-Generative Training for Set data}
\end{onehalfspace}

\bigskip{}

\noindent \emph{Author:} Bc. Jakub Bureš

\bigskip{}

\noindent \emph{Abstract:} This Master Thesis deals with hybrid discriminative and generative modeling and its possible utilization in multiple--instance learning. At first, technicalities of this approach are introduced; consequently a simple experiment is performed and lastly multi--instance learning itself is taken care of. The main aim of this work is to use the HMill framework with the Mill.jl library and involve contrastive learning into it, where the benefits of this approach are shown. 

\bigskip{}

\noindent \emph{Key words:} hybrid discriminative and generative modeling, multiple instance learning

\newpage{}

\pagestyle{plain}

\tableofcontents{}

\newpage{}
\pagestyle{headings}